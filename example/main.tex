\documentclass[
    11pt,
    largemargins,
    twoside,
    template_root=../,
    reference_icons=./Icons,
]{../BOOK}

\begin{document}

    \COVER[./cover.jpg]

    \TOC

    \CHAPTER[./combinations.jpg]{آنالیز ترکیبی}{
    آنچه در این فصل مورد بحث قرار خواهد گرفت،
    مبحث شمارش است که به محاسبه‌ی تعداد حالات رخداد یک پدیده،
    بدون بررسی تک تک حالات می‌پردازد.
    از کاربردهای این فصل می‌توان به
    محاسبه‌ی احتمالات پیش‌آمد‌ها،
    تخمین زمان اجرا و منابع مصرفی برنامه‌ها،
    برخی از تحلیل‌ها در گراف
    و ...
    اشاره کرد.
    }{
        بی‌نام ناشناس
    }

        \SECTION{اصل شمول و عدم شمول}
            \TARGET{اصل شمول و عدم شمول}
            \p
            همانطور که در توضیحات مربوط به اصل جمع نیز گفته شد، آن اصل
            فقط زمانی قابل استفاده است که حالات مختلف انجام یک عمل از دو مسیر،
            اشتراکی نداشته باشند. این اصل برای رفع این محدودیت ارائه شده است.
            منطق این اصل بسیار ساده است. اگر حالتی از انجام کار، در دو مسیر مشترک باشد،
            اگر از اصل جمع استفاده کنیم، این حالت دو بار شمرده می‌شود. برای حل این ضعف،
            به سادگی، این تعداد را یکبار از نتیجه کل کم می‌کنیم تا به تعداد حالات یکتا برسیم.

            \begin{DEFINITION}
                \p
                اگر بتوان فضای حالات عملی
                (مانند P)
                را به دو فضای
                $A_1$
                و
                $A_2$
                تقسیم کرد به نحوی که این دو فضا امکان اشتراک در اعضایشان را داشته باشند،
                آنگاه طبق \FOCUSEDON{اصل شمول و عدم شمول} تعداد اعضای فضای حالت کل برابر است با:
                $$|A_1 \cup	A_2| = |A_1| + |A_2| - |A_1 \cap A_2|$$
            \end{DEFINITION}

            \NOTE{
                اصل فوق قابلیت تعمیم دارد.
            }

            \begin{THEOREM}
                \p
                \FOCUSEDON{تعمیم اصل شمول و عدم شمول}
                را می‌توان به شکل زیر نوشت:
                \begin{align*}
                    |\bigcup\limits_{i=1}^n A_i| &= \sum\limits_{k=1}^n (-1)^{k+1} (\sum\limits_{1 \leq i_1 < \dots < i_k \leq n} |\bigcap\limits_{j \in \{i_1,...,i_k\}} A_j|) \\
                    &= \sum\limits_{1 \leq i_1 < \leq n} |A_{i_1}| \\
                    &- \sum\limits_{1 \leq i_1 < < i_2 \leq n} |A_{i_1} \cap A_{i_2}| \\
                    &+ \sum\limits_{1 \leq i_1 < i_2 < i_3 \leq n} |A_{i_1} \cap A_{i_2} \cap A_{i_3}| \\
                    &- \cdots
                \end{align*}
            \end{THEOREM}

            \begin{PROBLEM}[تعداد رشته‌های باینری]
                \p
                چه تعداد رشته باینری به طول ۸ وجود دارد که یا با ۱ آغاز شود و یا با ۰۰ به پایان برسد؟
            
                \SOLUTION[\REF{R} پاسخ از طریق اصل شمول و عدم شمول]{
                    \p
                    اگر تعداد رشته‌هایی که با ۱ آغاز می‌شوند را با
                    $A_{1}$
                    نشان دهیم، داریم (یک حالت برای بیت اول و ۲ حالت برای هر یک از ۷ بیت دیگر):
                    $$|A_1| = 1 \times 2^7$$
                    \p
                    اگر تعداد رشته‌هایی که با ۰۰ به پایان می‌رسند را با 
                    $A_{0}$
                    نشان دهیم، داریم
                    (یک حالت برای دو بیت آخر و ۲ حالت برای هر یک از ۶ بیت دیگر) :
                    $$|A_0| = 1^2 \times 2^6$$
                    \p
                    تعداد رشته‌هایی که با ۱ آغاز می‌شوند و با ۰۰ به پایان می‌رسند
                    (یک حالت برای بیت اول و دو بیت آخر و ۲ حالت برای هر یک از ۵ بیت دیگر) :
                    $$|A_1 \cap	A_0| = 1^3 \times 2^5$$
                    \p
                    بنابر اصل شمول و عدم شمول داریم :
                    $$|A_1 \cup	A_0| = |A_1| + |A_0| - |A_1 \cap	A_0| = 2^7 + 2^6 - 2^5$$
                }
            \end{PROBLEM}

            \SUBSECTION{پریش}

                \begin{DEFINITION}
                    \p
                    به هر جایگشتی از یک دنباله متناهی به نحوی که هیچ یک از اعضا در
                    جایگاه اصلی خود قرار نگیرند،
                    \FOCUSEDON{پریش}
                    گفته می‌شود.
                \end{DEFINITION}

                \p
                برای نمونه،
                دنباله حروف
                $TKSAR$
                یک پریش برای دنباله
                $STARK$
                می‌باشد.

                \begin{EXTRA}{بیشتر بدانید: معادلات سیاله}
                    \p
                    \FOCUSEDON{معادله سیاله}
                    در ریاضیات،
                    معادله‌ای چند جمله‌ای با متغیرهای صحیح
                    (مجهولات فقط می‌توانند مقادیر صحیح اتخاذ کنند)
                    است. شکل کلی این معادلات را می‌توان به شکل زیر نمایش داد
                    که در آن، تنها ‌$x_i$ها مجهول هستند
                    (ضرایب و توان‌ها می‌توانند هر مقداری حقیقی داشته باشند): 
                    $$\sum\limits_{i=1}^{n} \sum\limits_{j=1}^{d_i} ({a_{i,j}} \times {x_i}^{j}) = s$$
                \end{EXTRA}

            \SUBSECTION{کد}
                \p
                همچنین خوب است اگر به کد زیر نیز توجه کنید:
                \CODE{code.cpp}

        \newpage
        \SECTION{مسائل}

            \EPROBLEM{
                \p
                سه مهره رخ متمایز و صفحه شطرنجی
                $8\times8$
                داریم. به چند روش می‌توان این سه مهره را در سه خانه
                از این صفحه قرار داد به طوری که حداقل یک مهره
                وجود داشته باشد که توسط هیچ مهره‌ای تهدید نمی‌شود؟

                \EWSOLUTION{
                    -  کل حالات:
                
                        \[64\times63\times62\]
                
                    -  حالات نامطلوب: حالاتی که همه رخ‌ها تهدید بشوند.
                    
                        \[64\times7\times20\times2\]
                
                    -  حالات مطلوب: طبق اصل متمم برابر است با:
                
                        \[64\times63\times62 - 64\times14\times20\]
                }

                \ESOLUTION{
                    -  کل حالات: به دلیل تمایز رخ‌ها برابر است با:

                        \[ P(64,3) = 64\times63\times62\]

                    -  حالات نامطلوب: حالاتی که همه رخ‌ها تهدید شوند.
                    دو حالت داریم:
                    \begin{enumerate}
                        \item
                        رخ اول رخ دوم را تهدید کند:
                        \[64\times14\times20\]

                        \item
                        رخ اول رخ دوم را تهدید نکند:
                        \[64\times49\times2\]
                        
                    \end{enumerate}

                    -  حالات مطلوب طبق 
                    \CROSSREF[اصل متمم]{اصل شمول و عدم شمول}
                    برابر است با:
                        \[64\times63\times62\ - (64\times14\times20\ + 64\times49\times2\ )\]
                }
            }

        \newpage
        \SECTION{تمرینات}

            \EXERCISE
            سوال اول

            \EXERCISE
            سوال دوم

            \EXERCISE
            سوال سوم



        \CHAPTER{منابع}{در این بخش می‌توانید معنای علائم اختصاری منابع را ملاحظه کنید.}{?}

        \LRTSOURCELINE{R}{
            Kenneth H. Rosen, Discrete Mathematics and Its Applications, 7th Edition (1969)
        }
        
\end{document}
